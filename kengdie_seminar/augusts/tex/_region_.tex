\message{ !name(august_源码.tex)}%\documentclass[handout]{beamer}
\documentclass{article} \usepackage{amsmath}
\usepackage[svgnames]{xcolor} \usepackage{graphicx}
\usepackage{listings} \usepackage{xeCJK}
% -------language and font set-----
\setCJKmainfont{WenQuanYi Zen Hei}
% -------CODE input----------------
\lstset{
  %% Settings for listings package.
  % language=[ANSI]{C},
  language=C++, backgroundcolor=\color{Lavender},
  basicstyle=\footnotesize, breakatwhitespace=false, breaklines=true,
  captionpos=b, commentstyle=\color{olive},
  directivestyle=\color{blue}, extendedchars=true,
  frame=single,%shadowbox
  framerule=0pt, keywordstyle=\color{blue}\bfseries,
  morekeywords={*,define,*,include...,enumStateType}, numbersep=5pt,
  % rulesepcolor=\color{red!20!green!20!blue!20},
  showspaces=false, showstringspaces=false, showtabs=false,
  stepnumber=2, stringstyle=\color{purple}, tabsize=4, title=\lstname
}
% -----------margin set------------
\oddsidemargin=0pt \textwidth=440pt \marginparwidth=0pt
% --------Length set ------------
\setlength{\parindent}{0em} \setlength{\baselineskip}{1.8em}
\setlength{\parskip}{1ex}
% --------beamer style set --------- \usepackage{beamerthemeshadow}
% \usetheme{Frankfurt} \setbeamercolor{normal
% text}{bg=yellow!80!green} %background color using xcolor
% \setbeamertemplate{navigation symbols}{} %no navigation bar
% \setbeamertemplate{items}[ball]
% \hypersetup{colorlinks=true,linkcolor=red}

% -----------------those length are set specially for chinese
% \usepackage{beamerthemeshadow} \usetheme{Frankfurt}
% \setbeamercolor{normal
% text}{bg=yellow!80!green} %background color using xcolor
% \setbeamertemplate{navigation symbols}{} %no navigation bar
% \setbeamertemplate{items}[ball]
% \hypersetup{colorlinks=true,linkcolor=red}
\begin{document}

\message{ !name(august_源码.tex) !offset(385) }
\subsection{X\_igenetic\_probs.pbl}
基因间区参数比较简单,都是与GC content相关的,主要分为两部分:
\begin{itemize}
\item 多联体经验分布
\item 四阶马尔科夫转移概率\(1024X1的向量\)
\end{itemize}	
\begin{lstlisting}
  [1]
  #第一项是单碱基碱基组成
  # (a,c,g,t)= (0.29, 0.21, 0.21, 0.29)
  #第二项是多联体组成
  # Probabilities file for the intergenic region model
  # k = 4 最长多联体的长度(从零开始)
  # The P_l's
  [P_ls]
  # l=0
  # Values
  A   0.283
  C   0.217
  G   0.217
  T   0.283
  # l=1
  # Values
  AA  0.0916
  AC  0.0506
  AG  0.0717
  AT  0.0693
  CA  0.0728
  CC  0.0586
  CG  0.0137
  CT  0.0717
  GA  0.0596
  GC  0.048
  GG  0.0586
  GT  0.0506
  TA  0.0592
  TC  0.0596
  TG  0.0728
  TT  0.0916
  ……

  TTTGT   0.00223
  TTTTA   0.00303
  TTTTC   0.00243
  TTTTG   0.00256
  TTTTT   0.00707
  ……
  [EMISSION]
  # Vector size (4^(k+1))= 1024
  # Probabilities
  # 这里四个一组,和为1.
  AAAAA   0.3976
  AAAAC   0.1709
  AAAAG   0.1564
  AAAAT   0.2752
  AAACA   0.389
  AAACC   0.1645
  AAACG   0.1778
  ……
  # EOF
\end{lstlisting}
% ------------------外显子参数文件-----------
\message{ !name(august_源码.tex) !offset(779) }

\end{document} 
