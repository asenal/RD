\documentclass{beamer}
\usepackage{graphicx}
\usepackage{xeCJK}
\setCJKmainfont{WenQuanYi Zen Hei}
\usetheme{Frankfurt}
\useinnertheme{circles}
\setbeamercolor{normal text}{bg=yellow!80!green} %background color using xcolor
%\setbeamertemplate{navigation symbols}{}  %no navigation bar
\hypersetup{colorlinks=true,linkcolor=red}
\begin{document}
%%-----------------------------------
{
\setbeamertemplate{background canvas}{\includegraphics[width=\paperwidth,height=\paperheight]{model.jpg}}
\begin{frame}
    \title{分子进化算法介绍}
    \author{元件注释\\2011-09-07 }
    \institute{BGI-RD\\深圳}
    \maketitle
\end{frame}
}
\begin{frame}\frametitle{Outline}                       %contents only page !
\tableofcontents
\end{frame}
%%-------------------------
\section{背景介绍}
\begin{frame}\frametitle{背景介绍}
分子进化是在分子水平上对进化过程进行研究,始于20世纪60年代。分子水平的进化研究致力于两大问题:重建物种件的进化关系以及了解进化过程的动力于机制。前者属于系统学领域,后者是通过估计核苷酸置换速率以及采用序列数据来检测突变与选择模型进行研究[Yang]。\\
本次组会介绍如何从比对数据出发,得到物种间完整的进化关系,以及对结果的评估。涉及的概念和方法:
\begin{itemize}
\item 二叉树
\item 简约规则,Sankoff 算法
\item 碱基替换矩阵,进化距离估计
\item 最大似然规则,Felsenstein pruning 算法
\item 树的检验和模型比较:Bootstrap  LRT
\end{itemize}
\end{frame}

%%-----------------------
\section{树}
\begin{frame}
\frametitle{树}
物种(基因)间的谱系关系可以用一棵树表示,叶子节点代表可以观测的物种,内枝节点代表已灭绝的祖先,枝长通常是某种距离的测度。几乎所有进化关系都可以用二叉树表示(严格的星状树存在争议,另外,星状结构可以分解成内枝很短的几棵子树)。\\
不同的建树方法可以产生有根树或无根树,其中分子钟的存在是一个关键的假设,通常这一假设仅在近缘物种间成立,对不能确定树根的情况,通常引入关系更远的物种作为外群来置根。\\
\end{frame}
\begin{frame}
\frametitle{Newick记法}
$$(A:l1,(B:l2,C:l3))$$
用一对括号将临近物种归并到一个分支,冒号后的树脂是距离测度。
\end{frame}
\begin{frame}
\frametitle{树的计数}
\begin{itemize}
\item 无根树:$T_n=T_{n-1}*(2n-5)=3*5*7...(2n-5)=(2n-5)!!/2$
\item 有根树:$RT_n=(2n-3)*T_n$
\end{itemize}
数的增长大于指数速度。\\
\begin{itemize}
\item T3=1
\item T5=15
\item ......
\item T10=2027025
\end{itemize}
\end{frame}

\begin{frame}{树的搜索}
\begin{itemize}
    \item 添加法:$N=3+5+...+(2N-5)$
    \begin{itemize}
    		\item 不同阶段的分值不能比较
    \end{itemize}
    \item 星状分解法:$N=n(n-1)/2+(n-1)(n-2)/2+...+3=n(n^2-1)/6-7$
    \begin{itemize}
        \item 每步树的规模相同,分支可以比较
        \item 问题:星状树的似然值是怎么算的?这时候枝长还是进化距离的测度吗?
    \end{itemize}
		\item 分枝交换
\end{itemize}
\end{frame}

%---------------------------------
\section{确定树的分值计算}
\subsection{简约法}
\begin{frame}{简约规则}
简约规则选择同通过最少变化达到当前状态的树最为最优树。进化历程在这样的树上有最简洁的表现。\\
假定四个物种1,2,3,4;\\
三棵无根树:\\
\begin{itemize}
	\item T1:((1,2),(3,4)) 支持的位点构型:xxyy
	\item T2:((1,3),(2,4)) 支持的位点构型:xyxy
	\item T3:((1,4),(2,3)) 支持的位点构型:xyyx
	\item T*:((*,*),(*,*)) 非信息位点
\end{itemize}
\end{frame}

\begin{frame}{Sankoff算法}
先计算子树的局部最优分值,逐层向上,直到根节点。\\
约定罚分规则如下:\\
\begin{tabular}{c|c|c|c|c}
碱基&T&C&A&G\\
\hline
T&0&1&1.5&1.5\\
\hline
C&1&0&1&1.5\\
\hline
A&1.5&1&0&1\\
\hline
G&1.5&1.5&1&0\\
\end{tabular}
\end{frame}
\begin{frame}{简约法的缺点}
\begin{itemize}
	\item 计算过程枝使用树的结构信息,没有利用枝长
	\item 外枝显著大于內枝时倾向于将两个长枝聚合,‘长枝吸引’
\end{itemize}
这两个缺点反映的是同一个问题:不能处理多重击中。下面的ML模型解决这个问题。
\begin{itemize}
	\item 简约法需不需要计算两条序列间的距离?
	\item 有没有与距离相关的假定?
	\item 如果简约法估计有偏差,这偏差是向上还是向下的?
\end{itemize}	
\end{frame}
%%----------------------------
\subsection{似然法}
\begin{frame}{距离定义}
‘距离’是为了配合进化树的枝长而提出的形象说法,实际上我们真正关心的是分化时间。如果假定分化速率不变,那么二者没有区别。
在似然模型中,{\color{red}分化速率不变}是最重要的假设。\\
进化分析中,通常是根据比对数据估计分化距离(时间)。\\								
最简单的距离定义是位点差异比例,序列相似度越大,认为分化时间越短。\\
\begin{itemize}
	\item 认为进化时间与位点差异是线性关系
	\item 分化初期这种定义勉强靠谱
	\item 位点经过多次变化的,称为‘多重击中',这些位点破坏了线性关系
	\item 位点差异不是时间的线性函数,不能直接作为距离的度量
	\item 位点差异距离是对真实分化时间的高估/低估?
\end{itemize}	
{\color{red}{利用序列差异估计分化距离,难点是序列差异不是分化时间的线性函数,需要处理‘多重击中’}}\\
下面用马尔科夫链周全地解决多重击中的问题。								
\end{frame}
	
\begin{frame}{JC-69矩阵}
马尔科夫链第一要素:每一次变化只与当前状态有关。\\
这种关系是通过状态转移矩阵定义的,在当前语境,应该叫氨基酸替换速率矩阵。								
\begin{tabular}{c|c|c|c|c}
碱基&A&C&G&T\\
\hline
A&-3a&a&a&a\\
\hline
C&a&-3a&a&a\\
\hline
G&a&a-3a&a\\
\hline
T&a&a&a&-3a\\
\end{tabular}
这是最简单的替换矩阵,只有1个参数a,假设在分化过程中每个碱基以等速率向其它三种碱基转移。\\
\end{frame}

\begin{frame}{一点点计算}
以碱基x为例,说明M链怎样处理多重击中\\
在0时刻,x变异,其比例设为100%								
在t时刻,假如x比例为$p(t)$,则在下一个$t+1$时刻,
$$p(t+1)=p(t)*(1-3a)+(1-p(t)*a$$
这是个离散过程,但是因为一个单位时间(比如定义一个繁殖世代,这取决于a的大小)相对于整个进化历史非常短,因此可以近似成一个连续的过程。
$$p^{'}(t)=-4a*p(t)+a$$
解此微分方程:$p(t)=\frac{1}{4}+c*e^{-4at}$,利用边界条件$p(0)=1$确定$c=3/4$,所以:
$$p(t)=\frac{1}{4}+\frac{3}{4}*e^{-4at}$$
尽管有以上结果,但是分化时间t仍然是不能直接得到的。\\
因为每个碱基有三个变化方向,所以{\color{red}定义$d=3at$},差异位点比例$p^{*}=x/n$作为对$[1-p(t)]$的估计,有:
$$x/n=3/4-3/4*e^{-4d/3}$$
$$d=-3/4*log(1-3/4*x/n)$$
\end{frame}

\begin{frame}{一个位点的计算}
$$f(x_i|\tau_j,v_j,\theta)=\sum_y[\pi_{y_{root}}(\prod_{k=1}^{s}p_{y_{\sigma(k)},x_{ik}}(v_k,\theta)(\prod_{k=s+1}^{2s-2}p_{y_{i\sigma(k)}}(v_k,\theta))]$$
\end{frame}
%%----------------------------
\begin{frame}
\Huge {谢谢}
\end{frame}
%----------------------------------
\end{document}
