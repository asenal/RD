\documentclass{article}
\usepackage{xeCJK}
\begin{document}
\setCJKmainfont{WenQuanYi Zen Hei}
\author{于秋林}
\date{\today}
\title{项目日志:Hmm基本算法}
\maketitle
\abstract{
				这是2011年在深圳华大,元件注释小组的工作。6月开始,中途搁置3个月,10月利用周末时间完成。主要内容参考了崔晓源的博客‘我爱自然语言处理’。HMM适应能力非常强,可以胜任很多信息缺损情况下的字符处理,模式发现任务,但目前只是简单的一阶离散HMM模型Perl版本的实现,没有针对特定问题做定制,只做学习模型之用。
}
\newpage
\tableofcontents				
\newpage
\section{背景概述}
hi
\section{Perl实现}
ha
\section{用例测试}
hi
\section{收敛特征讨论}
hi
\end{document}				
